\hypertarget{GUI_SecGUI}{}\section{Graphical User Interface}\label{GUI_SecGUI}
ReMoDy can be compiled in two versions: with and without a graphical user interface (GUI). The former is used to visualize the simulation progress on a small set of data, while the latter is to perform larger-scale simulations where the concurrent visualization is not possible.

After the GUI-enabled version (view) is started, a display window pops up, showing the computational domain. The initial parameters of the scene can be set in the config file (see Sec. \hyperlink{Configfile_SecConfig}{Configuration File}). The configfile can be reloaded by making the display window active and hitting the 'l' key for 'load'. The keyboard control of the simulation has two modes: {\em keystroke\/}, and {\em commandline\/}. This is similar to vi text editor in UNIX. Initially the keystroke mode is active.

\begin{enumerate}
\item In the keystroke model the single-key-strokes can be used while the display window is active. These include:\end{enumerate}


\begin{itemize}
\item {\bf ESC}: exit. The same as 'q';\item {\bf ?}: show help.\item {\bf :}: switch to a terminal command mode (similar to vi). The same as '.';\item {\bf a}: show/hide the axes.\item {\bf b}: show/hide boundary faces of the mesh (not used when mesh is absent).\item {\bf c}: switch the color-scheme.\item {\bf f}: show the box frame.\item {\bf G}: show all the mesh edges.\item {\bf g}: show only the boundary mesh edges.\item {\bf l}: read configuration file.\item {\bf m}: display the menue on the console.\item {\bf N}: show mesh nodes.\item {\bf n}: show boundary vertexes.\item {\bf r}: start/stop the continuous run.\item {\bf s}: perform one step of the simulation. Analogous to '+'.\item {\bf v}: show boundary vectors.\item {\bf w}: dump current window into a xwd file.\item {\bf Z}: zoom-in view.\item {\bf z}: zoom-out view.\end{itemize}


\begin{enumerate}
\item The command mode is switched from the keystroke mode by hitting the ':' or '.' keys. To switch back from the command mode one has to enter an empty command, that is, press ENTER, while the command line is empty. In the command mode, the command line appears in the terminal windows from which the program has started. The commands one can enter in the command line include:\end{enumerate}


\begin{itemize}
\item {\bf bg}: go to background run mode. During this mode the all controls are disabled.\item {\bf cs}: switch the color scheme.\item {\bf fg}: go to foreground mode.\item {\bf ?, h, or help}: show help.\item {\bf r}: run a number of iterations.\item {\bf sp}: toggle between spheres and points in particle (molecule) display.\item {\bf wd}: start/stop window dump every fixed number of iterations.\item {\bf dw}: dump current window picture into a xwd file.\item {\bf e}: exit the program. Also saves current data.\item {\bf q}: quit without saving.\item {\bf .}: quit the command mode. Also hitting ENTER with empty command line. \end{itemize}
