\begin{Desc}
\item[Version:]1.0 \end{Desc}
\begin{Desc}
\item[Author:]Andrei Smirnov \href{mailto:andrei.v.smirnov@gmail.com}{\tt andrei.v.smirnov@gmail.com}\end{Desc}
\hypertarget{index_SecCont}{}\section{Contents}\label{index_SecCont}
\begin{itemize}
\item \hyperlink{index_SecIntro}{Introduction}\item \hyperlink{Model_SecModel}{Physical Model}\item \hyperlink{index_SecInstall}{Installation and Execution}\item \hyperlink{index_SecSetup}{Setup}\item \hyperlink{Configfile_SecConfig}{Configuration File}\item \hyperlink{GUI_SecGUI}{Graphical User Interface}\item \hyperlink{Output_SecOutput}{Output Procedures}\item \hyperlink{Implementation_SecImpl}{Implementation}\item \hyperlink{Examples_SecExamples}{Examples}\item \hyperlink{FAQ}{Frequently Asked Questions}\item \hyperlink{index_SecMisc}{Licencing and Support}\end{itemize}




 \hypertarget{index_SecIntro}{}\section{Introduction}\label{index_SecIntro}
This is the documentation of the ReMoDy (Reactive Molecular Dynamics) program, which implements the model of reactive molecular dynamics based on the \href{http://en.wikipedia.org/wiki/Collision_theory}{\tt Collision Theory}.



 \hypertarget{index_SecInstall}{}\section{Installation and Execution}\label{index_SecInstall}
\hypertarget{index_step1}{}\subsection{Step 1: Extracting the archive}\label{index_step1}
If you have the archive file, like remody.tbz (tar-bzipped) or remody.tgz (tar-gzipped) then files can be retrieved into a current directory as: \par
\par
{\bf  tar cvjf remody.tbz \par
\par
} or \par
\par
{\bf  tar cvzf remody.tgz \par
\par
} respectively.\hypertarget{index_step2}{}\subsection{Step 2: Compiling}\label{index_step2}
To compile the executable on Linux run make from the remody root directory, where the makefile was saved. Note that the subdirectories src/, run/, and obj/ should be also present.\hypertarget{index_step3}{}\subsection{Step 3: Running}\label{index_step3}
The executables are saved in the run/ directory. It should also contain the example remody.xml and demo.xml files, as well as an initial empty input file empty.dat.gz. To run the program with an OpenGL window active (slow, good for debugging), one can use the command: \par
\par
{\bf  ./view -f demo.xml empty.dat.gz }\par
\par
 This will start the program with the initial parameters read from the demo.xml file and the initial data read from empty.dat.gz file. On startup the program will open the window. One can point at the window with the mouse and press 'f' key to show the frame and 'r' key to run the simulation. Alternatively, one can use 's' key to run iterations step-by-step. Other key functions are described by pressing the '?' key.

To run the program in a batch mode without OpenGL output, one can use this command: \par
\par
{\bf  ./job -f demo.xml empty.dat.gz \& }\par
\par
 This will start the run. To change the parameters of the job, one should modify the input xml file accordingly (demo.xml, remody.xml, etc.). A link with the name remody is also set to one of the executables, like \par
\par
{\bf  ln -s ./job remody }\par
\par




 \hypertarget{index_SecSetup}{}\section{Setup}\label{index_SecSetup}
Both view (OpenGL based) and job (batch mode) executables read the configuration xml file. By default the file will have the same name as the executable, for example: job.xml or view.xml. This can be overwritten with the '-f' option. Two example files \href{../../run/remody.xml}{\tt remody.xml} and \href{../../run/demo.xml}{\tt demo.xml} are provided with the distribution.

The format of the xml file is in most cases self-explanatory with explanations provided in the \char`\"{}title\char`\"{} fields and comments. The main sections include: run, species, domain, and gui.

The {\bf run} section specifies the number of iterations, time steps, output intervals, and the maximum number of molecules to be used.

The {\bf species} section provides the list of species, and reactions between them.

The {\bf domain} section provides the specifications of the boundary conditions and the surface reactions and species.

The {\bf gui} section specifies the parameters used in the OpenGL output window, such as colors, line width, etc.

See also Section \hyperlink{Configfile_SecConfig}{Configuration File}



 \hypertarget{index_SecMisc}{}\section{Licencing and Support}\label{index_SecMisc}
The rest of this documentation provides the descriptioin of the structure and functionality of the classes, namespaces, and files of the code.

The code is provided under the \href{http://www.gnu.org/licenses/gpl.html}{\tt GPL licence}. Any suggestions/requests can be directed to the developers team at \href{http://nift.wvu.edu}{\tt NIFT}. 