\hypertarget{Configfile_SecConfig}{}\section{Configuration File}\label{Configfile_SecConfig}
\hypertarget{Configfile_SSecInvocation}{}\subsection{Invocation}\label{Configfile_SSecInvocation}
The config file is in XML format. It is loaded at program startup and contains the specification of all the main parameters, grouped in respective sections. An example \href{../syngas.xml}{\tt syngas.xml} file is provided which includes the setup of syngas reaction on the anode side of a high-temperature solid-oxide fuel cell.

The default name of the config file is $<$progname$>$.xml, where $<$progname$>$ is the name of the executable, such as {\tt view}, {\tt job}, or {\tt remody}. One can override the default name by using -f command line option, like: \par
\par
{\bf  remody -f syngas.xml }\par
\par
\hypertarget{Configfile_SSecSections}{}\subsection{Sections}\label{Configfile_SSecSections}
The sections are enclosed in XML tag pairs, like $<$tag$>$...$<$/tag$>$. where the tags are as follows:

\begin{itemize}
\item {\tt iterations}: specifies the integer number of iterations the code should run\item {\tt time}: several parameters of physical execution time in seconds, such as {\tt start}=start time, {\tt end}=end time, and {\tt step}=time step. The code will run until either the number of iterations as specified in the {\tt iterations} section or the end time is reached.\item {\tt molecules}: integer number equal to the maximum number of molecules that the program will be able to handle. It will determine the memory allocated at the beginning of execution.\item {\tt species}: this section lists all chemical species and reactions between them. Correspondingly, each specie is described in {\tt specie} section as:\end{itemize}
\begin{enumerate}
\item {\tt specie}: includes such parameters as {\tt mass} in Atomic units \mbox{[}au\mbox{]}, {\tt size} in nanometers \mbox{[}nm\mbox{]}, and specific heat, {\tt cp} in kJ/(mol$\ast$K). The {\tt specie} tag should also include an attribute {\tt id}, identifying a specie chemical formula, such as CO2, H2O, etc.\item {\tt reaction}: reaction tag has two attributes: {\tt reactants} and {\tt products}. The reactants tag should contain two species identifiers. Since only elementary (binary) reactions are considered, there should be exactly two identifiers for reactants. Each of the identifiers should correspond to one specie identifiers listed in the list of species. The products attribute should consist of one or two identifiers of reaction products. Also in the reaction section the following parameters are specified:\par
 {\tt activation} = reaction activation temperature in K, \par
 {\tt probability} = probability of reaction outcome as given by the {\tt products} attribute. This parameter should always be 1.0 if there is only one reaction with the given reactants. In case of several reactions with the same reactants, but with different products, this parameter should indicate the probability of this particular branch with given reaction products. \par
 {\tt enthalpy} = enthalpy of reaction in kJ/mol.\end{enumerate}
\begin{itemize}
\item {\tt domain}: The {\tt domain} section consists of the set of parameters describing the geometry and physical properties of the modeled media inside the computational domain, including:\end{itemize}
\begin{enumerate}
\item {\tt type}: specifies the geometry. Currently only {\tt box} type is supported.\item {\tt grid}: Specifies parameters of the rectangular grid used in segmentation algorithm for accelerating the interaction scheme. In particular, the {\tt cellsize} parameter determines the size of the grid cell. This size should be selected as small as possible but no less than twice the size of the largest specie. Decreasing the cell size will speed-up code execution in better than linear proportion of the cell-size (but no better than quadratic). At the same time it will increase memory utilization in proportion to it's 3-rd power. It is not recommended to increase the memory utilization above 90\%. One can use Unix 'top' utility to check memory utilization.\item {\tt energy}: Only used for Lennart-Jones type potentials, currently under development.\item {\tt bounds}: specifies spatial bounds of the computational domain as: xmin xmax ymin ymax zmin zmax.\item {\tt bulk}: this section specifies thermodynamic properties and gas composition inside the bulk of the domain. In particular, {\tt temperature} is given in Kelvin \mbox{[}K\mbox{]}, and for each specie, its {\tt density} is specified in \mbox{[}kg/m$^\wedge$3\mbox{]}.\item {\tt boundary}: each boundary of the domain contains the description of thermodynamic properties and gas composition on the other side of the boundary in the same format as for the bulk of the domain. In addition to these, the boundary tag should have the boundary identifier {\tt id}-attribute, such as \char`\"{}top\char`\"{}, \char`\"{}bottom\char`\"{}, \char`\"{}right\char`\"{}, \char`\"{}left\char`\"{}, \char`\"{}front\char`\"{}, and \char`\"{}rear\char`\"{}. Also additional boundary tag is {\tt type}, which can be one of: \char`\"{}open\char`\"{}, \char`\"{}elastic\char`\"{}, and \char`\"{}periodic\char`\"{}. In the case of open boundary the molecules can freely cross the boundary, in which case they will be removed from the domain. In case of elastic boundary the molecules will bounce from the boundary like from an elastic wall. For periodic boundaries, the molecules crossing the boundary will reappear from the opposite boundary. \hyperlink{structBoundary}{Boundary} description can also contain the list reactions, between the boundary species given in the same format as in the {\tt species} section.\end{enumerate}


\begin{itemize}
\item {\tt gui}: the gui section describes parameters related to graphical output used when running an OpenGL based version with a visual window output.\end{itemize}
\begin{enumerate}
\item {\tt translation}: initial translation of the scene.\item {\tt vector}: parameters for displaying vectors.\item {\tt frame}: parameters for displaying a domain frame.\item {\tt mesh}: parameters for displaying mesh. In particular the {\tt node} tag specifies the parameters for displaying particles, or molecules, such as using points or spheres ({\tt type}), etc. Note that using 'spehere' for type may significantly slow-down simulation in GUI mode.\end{enumerate}


\begin{itemize}
\item {\tt xterm}: if set to 1 this parameter will force terminal dumping of certain parameters, like temperature, energy, number of molecules, and species concentrations after every time-step. This can be useful if the time-dependence of concentrations and other parameters needs to be retrieved. In the last case the output can be redirected to a file, which can be processed with the {\tt readlog} utility (see \hyperlink{FAQ_SecFAQ}{How do I} produce time-curves ...?\char`\"{}). \end{itemize}
